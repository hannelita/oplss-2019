%%%%%%%%%%%%%%%%%%%%%%%%%%%%%%%%%%%%%%%%%
% University/School Laboratory Report
% LaTeX Template
% Version 3.1 (25/3/14)
%
% This template has been downloaded from:
% http://www.LaTeXTemplates.com
%
% Original author:
% Linux and Unix Users Group at Virginia Tech Wiki 
% (https://vtluug.org/wiki/Example_LaTeX_chem_lab_report)
%
% License:
% CC BY-NC-SA 3.0 (http://creativecommons.org/licenses/by-nc-sa/3.0/)
%
%%%%%%%%%%%%%%%%%%%%%%%%%%%%%%%%%%%%%%%%%

%----------------------------------------------------------------------------------------
% PACKAGES AND DOCUMENT CONFIGURATIONS
%----------------------------------------------------------------------------------------

\documentclass{article}

\usepackage[version=3]{mhchem} % Package for chemical equation typesetting
\usepackage{siunitx} % Provides the \SI{}{} and \si{} command for typesetting SI units
\usepackage{graphicx} % Required for the inclusion of images
\usepackage{natbib} % Required to change bibliography style to APA
\usepackage{amsmath} % Required for some math elements 
\usepackage{amssymb}


\usepackage{proof}
\setlength{\inferLineSkip}{4pt}

\usepackage{hyperref}
\usepackage{dashrule}

\usepackage{bussproofs}




\setlength\parindent{0pt} % Removes all indentation from paragraphs

\renewcommand{\labelenumi}{\alph{enumi}.} % Make numbering in the enumerate environment by letter rather than number (e.g. section 6)

%\usepackage{times} % Uncomment to use the Times New Roman font

%----------------------------------------------------------------------------------------
% DOCUMENT INFORMATION
%----------------------------------------------------------------------------------------

\title{From Singleton to Linear Logic \\ Frank Pfenning \\ OPLSS 2019} % Title




\begin{document}
	
	\maketitle % Insert the title, author and date
	
	\begin{center}
		\begin{tabular}{l r}
			Date Performed: & June 18th 2019 \\ % Date the experiment was performed
			Students: & J.W.N. Paulus  \\
			& R. Gurdeep Singh \\ % Partner names
			& H. C. A. Tavante
		\end{tabular}
	\end{center}
	
	% If you wish to include an abstract, uncomment the lines below
	% \begin{abstract}
	% Abstract text
	% \end{abstract}
	
	%----------------------------------------------------------------------------------------
	% SECTION 1
	%----------------------------------------------------------------------------------------
	\setcounter{section}{3} % hack to have correct numbers
	\section{Lecture 4: Linear logic \(\otimes\) session types}
	\subsection{Connectives of Linear Logic}

	Judgement 
	\[
	\underbrace{\Gamma}_{A_1,...,A_n} \vdash A
	\]

	Id:

	\[
	\infer[\scriptsize{ID}]
	{A \vdash A}
	{}
	\]

	Cut-rule:
	\[
	\infer[\scriptsize{CUT}]
	{\Gamma_{1}, \Gamma_{2} \vdash C}
	{\Gamma_{1} \vdash A \hspace{2em} \Gamma_{2}, A \vdash C}
	\]

  $\oplus$ Right and Left rules (Internal choice)
	\[
	\infer[\oplus\scriptsize{R_1}]
	{\Gamma \vdash A \oplus B}{\Gamma \vdash A}
	\qquad
	\infer[\oplus\scriptsize{R_2}]
	{\Gamma \vdash A \oplus B}{\Gamma \vdash B}
	\qquad
	\infer[\oplus\scriptsize{L}]
	{\Gamma, A \oplus B \vdash C}{\Gamma, A \vdash C \hspace{1em} \Gamma, B \vdash C}
	\]


	$\&$ Right and Left rules (External choice)

	\[
	\infer[\&\scriptsize{R}]
	{\Gamma \vdash A \& B}{\Gamma \vdash A \hspace{1em} \Gamma \vdash B}
	\qquad
	\infer[\&\scriptsize{L_1}]
	{\Gamma, A \& B \vdash C}{\Gamma, A \vdash C}
	\qquad
	\infer[\&\scriptsize{L_2}]
	{\Gamma, A \& B \vdash C}{\Gamma, B \vdash C}
	\]


	$\otimes$ Right and Left rules (Tensor; dual choice)

	\[
	\infer[\otimes\scriptsize{R}]
	{\Gamma_{1}, \Gamma_{2} \vdash A \otimes B}{\Gamma_{1} \vdash A \hspace{1em} \Gamma_{2} \vdash B}
	\qquad
	\infer[\otimes\scriptsize{L}]
	{\Gamma , A \otimes B \vdash C}{\Gamma, A, B \vdash C}
	\]

	$\multimap$ Right and Left rules (Also kown as Linear implication or "Loly")

	\[
	\infer[\multimap\scriptsize{R}]
	{\Gamma \vdash A \multimap B}{\Gamma, A \vdash B}
	\qquad
	\infer[\multimap\scriptsize{L}]
	{\Gamma_{1}, \Gamma_{2}, A \multimap B \vdash C}{\Gamma_{1} \vdash A \hspace{1em} \Gamma_{2}, B \vdash C}
	\]

	No resources at all (the dot $ \bullet $ means there are no resources):

	\[
	\infer[\scriptsize{1R}]
	{\bullet \vdash 1}{}
	\qquad
	\infer[\scriptsize{1L}]
	{\Gamma, 1 \vdash C}{\Gamma \vdash C}
	\]


	\subsection{???}
	\subsection{???: Implementing a queue}
	
	
	
	
	%----------------------------------------------------------------------------------------
	% BIBLIOGRAPHY
	%----------------------------------------------------------------------------------------
	
	\bibliographystyle{apalike}
	
	\bibliography{sample}
	
	%----------------------------------------------------------------------------------------
	
	
\end{document}